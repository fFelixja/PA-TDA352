\documentclass{article}
\author{Felix Jansson \& Rikard Hjort}
\title{Programming Assignment}

\begin{document}
The personal number used is 199603177792.

\section{Running the assignment parts}
The assignments are all in different, clearly named Haskell files. To run them
all, use \texttt{cabal run} or run the \texttt{Main.hs} file using GHC. This
prints the output of all the separate assignment files (in the case of the first
assignment, runs the supplied tests). There is also a compiled binary, called
\texttt{Main}, for Linux x86-64 which can be run directly.

\section{A Math Library for Cryptography}
There is nothing out of the ordinary here. The extended Euclidian algorithm
follows the textbook implementation. The Euler $\phi$ function is implemented
naively, simply trying all integers below the input to see if the gcd is 1 or
not. The modular inverse uses the EEA to find the coefficients for Bezout's
identity. Fermat's primality test is also straightforward, but first of all
tries to invalidate a prime by using Fermat's theorem, and only then checks if
any of the tested numbers was a divisor.\footnote{We did find searching up to
  $\frac{n}{3}$ a bit curious. Why not stop at roughly $\sqrt{n}$, e.g., by
  multiplying the tested number by itself each time and comparing to $n$?} The
has collision probability is calculated with high precision, by checking the
probability of getting a collision at each sample point.

The library also includes some functions that were helpful in the other
assigments, especially \texttt{modN}, which uses exponentiation by squaring,
something that became necessary to solve the ElGamal assignment in reasonable time.

\section{Special Soundness of Fiat-Shamir sigma-protocol}
Knowing that the same nonce was used once for a challenge $c=0$, and once for
$c=1$, we can deduce the secret key.

Call the repeated nonce $r$. In the first step of the protocol, $R=r^2$ is sent
to the verifier. We thus find as situation when $R$ is repeated, and know that
$r$ was also repeated, with high probability. In one of the runs, the
challenge sent by the verifier was $c=0$, meaning the prover sent back $r$.
Thus, we know $r$; it is $s$, the final message of the run. We turn to the run
of the protocol that used the same $R$, and $c=1$. In this case, the final
message $s=rx$, $x$ being the secret key. Since we know $r$, we can easily
compute $r^{-1}$ using our \textit{modinv} function, and thus get $x=sr^{-1}$.

The function \textit{collectInfo} gives back the two runs that used the same
nonce. \textit{recoverX} then calculates the secret key, $x$.\\

Decoded message: A common mistake that people make when trying to design something completely foolproof is to underestimate the ingenuity of complete fools.
\section{Decrypting CBC with simple XOR}

The weakness in this particular cipher is the simplistic use of the key. By
using that me know the first message block, $m_0 = 199603177792$, that $IV =
6725DD9E6DE0$, and that $c_0 = k \oplus m_0 \oplus IV = 823C1EE8E02D$, we can
easily calcuclate $k = c_0 \oplus m_0 \oplus IV$. We then use the key to decrypt
the rest of the message, using a simple decryption circuit.\\

Decoded message: 199603177792Do or do not. There is not try. - Master Yoda000 \\

As you can see, the first block is the one that was previously known, and the
last block contains padding.

\section{Attacking RSA}
Since the same message has been encrypted by three different recipients and modulus we can solve $c$ the linear congruence using the chinese remainder theorem:\\
$c = c1 mod N1$\\
$c = c2 mod N2$\\
$c = c3 mod N3$\\

Since we know that all recipients used the same public key, 3, we know that $c=m^3$. Therefore we can recover the message simply by taking the cube root of $c$.\\

Decoded message: Taher ElGamal

\section{Attacking ElGamal}

Since we have a limit key space due to the weak random number generator, i.e. $1000$ different keys,  we can do an exhaustive search to find the used private key. By comparing c1 with $g^k$ and when $c1==g^k$ k is the private key.\\

Next we calculate the inverse of the public key s.t we have $g^-x$ using our $modInv$ function from CryptoLib, which returns the Bèzout coefficients.\\

With access to the private key $k$ and public key inverse we can now compute the shared secret which is $g^-xk$. Mulitple c2 with the shared secret we recover the the message.\\

Decoded message: The only fully secure symmetric cryptosystem is Bruce Schneier looking in a mirror.S

\end{document}

Outputs:

Fiat-Shamir
Recovered message: 4816288706411788595854482353325980769935445651749260003868792995194214307306594280862156886855814368607403917919844986650534525895093894287674151194995984150187677530211545894352340899476975335318931162357235019417898992136273817906183556642558478413269234989049712445020649839325080309123540571682414843048452532259143336583333132543738913553664838445187662546934069030259
Decoded message: A common mistake that people make when trying to design something completely foolproof is to underestimate the ingenuity of complete fools. - Douglas Adams›

CBC
Recovered message: "199603177792Do or do not. There is not try. - Master Yoda000"
